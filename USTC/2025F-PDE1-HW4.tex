\documentclass[UTF8,fontset=windows,12pt]{ctexart}
%\setcounter{tocdepth}{1}
%\setcounter{secnumdepth}{4}
%\usepackage{ctex} %若要输入中文请取消注释或者把article改成ctexart
% This first part of the file is called the PREAMBLE. It includes
% customizations and command definitions. The preamble is everything
% between \documentclass and \begin{document}.
%\usepackage[OT1]{fontenc}
%\usepackage{fontspec}
%\setmainfont{Times New Roman}
%\setCJKmainfont{Microsoft YaHei}
\usepackage[left=1.91cm,right=1.91cm,top=2.54cm,bottom=2.54cm]{geometry}
\usepackage{amsfonts}
\usepackage{amsmath}
\usepackage{amssymb}
\usepackage[upint,lcgreekalpha]{stix}
\usepackage[type1]{garamondlibre}
%\usepackage{esint}
\usepackage{pgfplots}
\pgfplotsset{compat=1.18}
\usepackage{tikz}
\usetikzlibrary{arrows}
\usepackage{extarrows}
\usepackage[only,llbracket,rrbracket]{stmaryrd}
\usepackage{titlesec}
\usepackage{bm}
\usepackage{graphicx}
\usepackage{appendix}
\usepackage{tikz-cd}
\usepackage{tikz-3dplot}
\usetikzlibrary{arrows.meta, decorations.pathreplacing, calc, patterns}
\usepackage{float}
\usepackage{mathtools}
\usepackage{amsthm}
\usepackage{verbatim}              % better theorem environments
\usepackage{setspace}
\usepackage{enumitem}
\setenumerate[1]{itemsep=0pt,partopsep=0pt,parsep=\parskip,topsep=5pt}
\setitemize[1]{itemsep=0pt,partopsep=0pt,parsep=\parskip,topsep=5pt}
\setdescription{itemsep=0pt,partopsep=0pt,parsep=\parskip,topsep=5pt}
\usepackage{xcolor}
\usepackage[colorlinks,linkcolor=black,anchorcolor=black,citecolor=black]{hyperref}
%\usepackage{apacite}
\usepackage[numbers]{natbib}
%\usepackage{showkeys}
\usepackage{cases}
\usepackage{fancyhdr}
\usepackage[scaled=0.92]{helvet}	% set Helvetica as the sans-serif font
%\usepackage{upgreek}
%\renewcommand{\rmdefault}{ptm}		% set Times as the default text font
%\usepackage{newtxtext}
\bibliographystyle{plain}
% If AMS-LaTeX is used, it can be loaded before or after mtpro2		
% The following loads metro and defines some common MTPro options [2, 4]
%\usepackage[lite,subscriptcorrection,nofontinfo]{mtpro2}
%\pdfmapfile{+mtpro2.map}

%\usepackage{txfonts}

% various theorems, numbered by section

\makeatletter
\renewenvironment{proof}[1][\proofname]{\par
  \pushQED{\qed}%
  \normalfont \topsep6\p@\@plus6\p@\relax
  \trivlist
  \item[\hskip\labelsep
        \bfseries % 关键修改:斜体改为加粗
        #1\@addpunct{.}]\ignorespaces
}{%
  \popQED\endtrivlist\@endpefalse
}
\makeatother
\newtheoremstyle{kaiti-theorem}   % 样式名称
  {3pt}                           % 上方间距
  {3pt}                           % 下方间距
  {\kaishu\upshape}               % 正文字体:中文楷体 + 英文直立
  {}                              % 缩进量
  {\bfseries\upshape}      % 头部字体(定理名):中文楷体粗体 + 英文直立粗体
  {.}                             % 定理名后标点
  {0.5em}                         % 定理名后间距
  {}                              % 头部说明(留空)

% 应用自定义样式
\theoremstyle{kaiti-theorem}
\newtheorem{thm}{定理}[section]
\newtheorem{nota}[thm]{记号}
\newtheorem{question}{问题}
\newtheorem*{questionn}{思考}
\newtheorem{exe}{习题}[section]
\newtheorem{assump}[thm]{假设}
\newtheorem*{thmm}{定理}
\newtheorem{defn}{定义}[section]
\newtheorem{lem}[thm]{引理}
\newtheorem*{lemm}{引理}
\newtheorem{prop}[thm]{命题}
\newtheorem*{propp}{命题}
\newtheorem*{claim}{断言}
\newtheorem{cor}[thm]{推论}
\newtheorem{conj}[thm]{猜想}
\newtheorem{rmk}{注记}[section]
\newtheorem{ex}{例}[section]
\newtheorem{pf}{证明}
\newtheorem{problem}{作业题}
\numberwithin{equation}{section}





\DeclareMathOperator{\id}{id}
\DeclareMathOperator*{\esssup}{ess\,sup}
\renewcommand{\Re}{\textbf{Re}\,}  
\renewcommand{\Im}{\textbf{Im}\,}  
\newcommand{\R}{\mathbb{R}}  
\newcommand{\FH}{\mathbb{H}}  
\newcommand{\C}{\mathbb{C}}  
\newcommand{\Z}{\mathbb{Z}}
\newcommand{\X}{\mathbb{X}}
\newcommand{\N}{\mathbb{N}}
\newcommand{\Q}{\mathbb{Q}}
\newcommand{\T}{\mathbb{T}}
\newcommand{\TT}{\mathcal{T}}
\newcommand{\TP}{\overline{\partial}{}}
\newcommand{\TJ}{\langle\TP\rangle}
\newcommand{\TL}{\overline{\Delta}{}}
\newcommand{\curl}{\text{curl }}
\newcommand{\dive}{\text{div }}
\newcommand{\bd}[1]{\mathbf{#1}}  % for bolding symbols
\newcommand{\bds}[1]{\boldsymbol{#1}}  % for bolding symbols
\newcommand{\RR}{\mathcal{R}}      % for Real numbers
\newcommand{\sh}{\mathcal{S}}  
\newcommand{\sph}{\mathbb{S}}  
\newcommand{\Om}{\Omega}
\newcommand{\OmT}{{\Omega_T}}
\newcommand{\ut}{{U_T}}
\newcommand{\vp}{\varphi}
\newcommand{\Th}{\Theta}
\newcommand{\Lam}{\Lambda}
\newcommand{\Omb}{{\overline{\Omega}}}
\newcommand{\OmbT}{{\overline{\Omega_T}}}
\newcommand{\ub}{{\overline{U}}}
\newcommand{\ubt}{{\overline{U_T}}}
\newcommand{\q}{\quad}
\newcommand{\p}{\partial}
\newcommand{\pb}{\boldsymbol{\partial}}
\newcommand{\lap}{\Delta}
\newcommand{\dd}{\mathfrak{D}}
\newcommand{\DD}{\mathcal{D}}
\newcommand{\den}{{\bar\rho}_{0}}
\newcommand{\Dt}{{\p}_{t}+v^{k}{\p}_{k}}
\newcommand{\col}[1]{\left[\begin{matrix} #1 \end{matrix} \right]}
\newcommand{\noi}{\noindent}
\newcommand{\nab}{\nabla}
\newcommand{\cnab}{\overline{\nab}}
\newcommand{\cp}{\overline{\partial}{}}
\newcommand{\dx}{\,\mathrm{d}x}
\newcommand{\dxx}{\,\mathrm{d}\bds{x}}
\newcommand{\xxi}{\boldsymbol{\xi}}
\newcommand{\eeta}{\boldsymbol{\eta}}
\newcommand{\dxxi}{\,\mathrm{d}\boldsymbol{\xi}}
\newcommand{\deeta}{\,\mathrm{d}\boldsymbol{\eta}}
\newcommand{\dxi}{\,\mathrm{d}\xi}
\newcommand{\deta}{\,\mathrm{d}\eta}
\newcommand{\dy}{\,\mathrm{d}y}
\newcommand{\dyy}{\,\mathrm{d}\bds{y}}
\newcommand{\dz}{\,\mathrm{d}z}
\newcommand{\dzz}{\,\mathrm{d}\bds{z}}
\newcommand{\dph}{\,\mathrm{d}\varphi}
\newcommand{\dt}{\,\mathrm{d}t}
\newcommand{\dr}{\,\mathrm{d}r}
\newcommand{\dtt}{\,\mathrm{d}\tau}
\newcommand{\dS}{\,\mathrm{d}S}
\newcommand{\dss}{\,\mathrm{d}\sigma}
\newcommand{\dmu}{\,\mathrm{d}\mu}
\newcommand{\dnu}{\,\mathrm{d}\nu}
\newcommand{\dV}{\,\mathrm{d}V}
\newcommand{\ds}{\,\mathrm{d}s}
\newcommand{\drr}{\,\mathrm{d}\rho}
\newcommand{\dist}{\text{dist }}
\newcommand{\vol}{\text{vol}\,}
\newcommand{\volo}{\text{vol}(\Omega)}
\newcommand{\spt}{\text{Spt}\,}
\newcommand{\Tr}{\text{Tr}\,}
\newcommand{\lee}{\langle}
\newcommand{\ree}{\rangle}
\newcommand{\xee}{\langle\xi\rangle}
\newcommand{\xxee}{\langle\boldsymbol{\xi}\rangle}
\newcommand{\xeta}{\langle\eta\rangle}
\newcommand{\xeeta}{\langle\boldsymbol{\eta}\rangle}
\newcommand{\kk}{\kappa}
\newcommand{\lam}{\lambda}
\newcommand{\io}{\int_{\Omega}}
\newcommand{\iu}{\int_{U}}
\newcommand{\is}{\int_{\Sigma}}
\newcommand{\ipo}{\int_{\partial\Omega}}
\newcommand{\ipu}{\int_{\partial U}}
\newcommand{\ig}{\int_{\Gamma}}
\newcommand{\ir}{\int_{\R}}
\newcommand{\ird}{\int_{\R^d}}
\newcommand{\irn}{\int_{\R^n}}
\newcommand{\fpi}{\frac{1}{(2\pi)^{\frac{d}{2}}}}
\newcommand{\ddt}{\frac{\mathrm{d}}{\mathrm{d}t}}
\newcommand{\ddx}{\frac{\mathrm{d}}{\mathrm{d}x}}
\newcommand{\eps}{\varepsilon}
\providecommand{\jump}[1]{\left\llbracket #1 \right\rrbracket }
\providecommand{\len}[1]{\lee #1 \ree }
\providecommand{\ino}[1]{\left\| #1 \right\| }
\providecommand{\bno}[1]{\left| #1 \right| }
\newcommand{\red}{\textcolor{red}}
\newcommand{\green}{\textcolor{green}}
\newcommand{\blue}{\textcolor{blue}}
\newcommand{\nnr}{{[n+1]}}
\newcommand{\nnn}{{[n]}}
\newcommand{\nnl}{{[n-1]}}
\newcommand{\nnll}{{[n-2]}}	
\newcommand{\mmr}{{[m+1]}}
\newcommand{\mmm}{{[m]}}
\newcommand{\mml}{{[m-1]}}
\newcommand{\mmll}{{[m-2]}}

%---------------Special variables--------------
\newcommand{\CC}{\mathfrak{C}}  
\newcommand{\NN}{\mathbf{N}}
\newcommand{\LH}{\mathcal{L}}
\newcommand{\HH}{\mathcal{H}}  
\newcommand{\EE}{\mathcal{E}}  
\newcommand{\FT}{\mathcal{F}}  
\newcommand{\fT}{\mathbf{T}}  
\newcommand{\FA}{\mathbf{A}}  
\newcommand{\MH}{\mathcal{M}}
\newcommand{\NH}{\mathcal{N}}
\newcommand{\PH}{\mathcal{P}}
\newcommand{\VH}{\mathcal{V}}
\newcommand{\XX}{\mathfrak{X}}
\newcommand{\xx}{{\bds{x}}}
\newcommand{\bx}{\mathbf{x}}
\newcommand{\by}{\mathbf{y}}
\newcommand{\bp}{\mathbf{p}}
\newcommand{\bq}{\mathbf{q}}
\newcommand{\pp}{{\bds{p}}}
\newcommand{\qq}{{\bds{q}}}
\newcommand{\uu}{\mathbf{u}}
\newcommand{\ww}{\mathbf{w}}
\newcommand{\vv}{\mathbf{v}}
\newcommand{\yy}{{\bds{y}}}
\newcommand{\zz}{{\bds{z}}}
\newcommand{\zo}{\mathbf{0}}
\newcommand{\ee}{\mathbf{e}}
\newcommand{\fa}{\mathbf{a}}
\newcommand{\fb}{\mathbf{b}}
\newcommand{\fc}{\mathbf{c}}
\newcommand{\ff}{\bds{f}}
\newcommand{\fr}{\mathbf{r}}
\newcommand{\fh}{\mathbf{h}}
\newcommand{\fm}{\mathbf{m}}
\newcommand{\fg}{\mathbf{g}}
\newcommand{\FF}{\mathbf{F}}
\newcommand{\GG}{\mathbf{G}}
\newcommand{\BB}{\mathbf{B}}



\begin{document}
\title{\LARGE{\bf  2025年秋季学期偏微分方程作业四}}
\author{极大值原理、格林函数}
\date{截止时间:2026年元月5日下课前}

\maketitle

\textbf{作业题1--9是必做题,作业题10是选做题。}

\begin{problem}[习题4.1.3]
设$f:\R\to\R$是单调递增函数, $\Omega\subset\R^d$是有界区域, 常数$T>0$. 设$u\in C_1^2(\Omega_T)\cap C(\overline{\Omega_T})$是如下方程的解
{\small\begin{equation*}
\begin{cases}
\p_t u-\lap u+f(u)=0,\quad& \text{in}\,\,\Omega_T\\
u(0,\xx)=\varphi(\xx),\q\q&t=0,~\xx\in\Omb\\
u(t,\xx)=g(t,\xx), \quad \quad& t\geq 0, \xx\in\p\Omega,
\end{cases}
\end{equation*}}
其中$\varphi, g$是给定的有界光滑函数。
\begin{itemize}
\item [(1)]  设$f\in C^1$, \textbf{叙述并证明}微分算子$\mathcal{L}u:=\p_t u-\lap u +f(u)$的比较原理。 
\item [(2)]  如果只假设$f$连续,证明上述方程解的唯一性。(提示:可以用能量法。)
\end{itemize}
\end{problem}

\begin{problem}[习题4.1.4]
考虑一维热方程$$u_t-u_{xx}=0\text{ in }[0,+\infty)\times (0,\pi),\q\q u(t,0)=u(t,\pi)=0~(t\geq 0).$$
\begin{enumerate}
\item [(1)] 证明: 对任意$a\in\mathbb{R}$, $v(t,x) = a e^{-t}\sin x$满足上述方程。

\item [(2)] 若初值选取为$u(0,x)={\footnotesize \begin{cases} x & x\in[0,\frac{\pi}{2}] \\ \pi-x & x \in [\frac{\pi}{2},\pi]\end{cases}}$, 证明:对应的解满足$e^{-t} \leq \max\limits_{x\in[0,\pi]}u(t,x) \leq \frac{\pi}{2} e^{-t}.$
\end{enumerate}

提示: (2)可以用比较原理,可以联想(1)中解的初值是什么。
\end{problem}


\begin{problem}[习题4.2.3,下调和函数的弱极值原理]
设$\Om\subset\R^d$是有界开集(未必是区域)。函数$u\in C^2(\Om)\cap C(\Omb)$是$\Om$中的\textbf{下调和函数}, 即$-\lap u\leq 0$在$\Om$内恒成立。
\begin{itemize}
\item [(1)] 证明:对任意球$B(\xx,r)\Subset \Om$,成立$u(\xx)\leq \fint_{B(\xx,r)}u(\yy)\dyy.$
\item [(2)] 证明:$\max\limits_{\overline{\Omega}} u=\max\limits_{\p\Omega}u$. 
\item [(3)] 证明:若$v\in C^3(\Om)\cap C^1(\Omb)$是$\Om$内的调和函数,则$|\nab v|$在$\p\Om$上达到最大值。
\end{itemize}

提示:(1)模仿调和函数平均值原理的证明。(2) 先证明$-\lap u<0$的情况,对一般情况考虑扰动$u_\eps(\xx)=u(\xx)+\eps|\xx|^2$. (3) 计算$\lap(|\nab v|^2)$并证明它非负。
\end{problem}

\begin{problem}[习题4.2.5,选做]
设$\Om:=B(\mathbf{0},1)\subset \R^d$, $u\in C^3(\Om)\cap C^1(\Omb)$满足
\[
\lap u-2u=1~~\text{ in }\Om,~~u|_{\p\Om}=g(\xx).
\] 这里$g$是$\p\Om$上的连续函数,证明:存在常数$C>0$使得$\max\limits_{\Omb}|u|\leq \max\limits_{\p\Om}|g|+C.$

提示:选取充分大的常数$\lam>0$, 使得$\lap (u^2+\frac{\lam}{4}|\xx|^2)\geq 0$.
\end{problem}

\begin{problem}[习题4.2.6]
设$u\in C^3(\Om)\cap C(\Omb)$是开集$\Om$内的\textbf{非负}调和函数, 球$B(\xx_0,R)\Subset \Om$.证明:对任意$1\leq i\leq d$有$|\p_{x_i} u(\xx_0)|\leq \dfrac{d}{R}u(\xx_0)$.

提示:$u\geqslant 0$表明$u=|u|$,思考在模仿命题4.2.9的证明过程中如何利用这个事实得到我们想要的结论。
\end{problem}

\begin{problem}[习题4.2.8]
设$u$是$\R^d$中的调和函数。证明:若$u$满足以下两条件中任一个,则$u$是常数。
\begin{itemize}
\item [(1)] $\exists\text{ 常数 } C>0,p>0$ 使得 $|u(\xx)|\leq C\left(\log(1+|\xx|^{p})+1\right)~~\forall \xx\in\R^d.$ 
\item [(2)] $\exists\text{ 常数 } C\in\R,~u(\xx)\geq C~~\forall \xx\in\R^d.$
\end{itemize}
\end{problem}

\begin{problem}[习题4.3.5]
设$u_i\in C^2(\Om)\cap C(\Omb)~(i=1,2)$是如下方程的解
\[
-\lap u_i+c_i(\xx)u_i=0\q\text{ in }\Om,\q\q u_i|_{\p\Om}=g_i(\xx).
\]若$c_2(\xx)\geq c_1(\xx)\geq 0$和$g_1(\xx)\geq g_2(\xx)\geq 0$恒成立,证明:$u_1(\xx)\geq u_2(\xx)$在$\Omb$上恒成立。

提示:先证明$u_i\geq 0$, 再考虑$u_1-u_2$满足的方程和边值条件。
\end{problem}


\begin{problem}[习题5.2.1]
设$\Om\subset\R^d~(d\geq 2)$是边界光滑的有界区域,定义$D:=\sup\{|\xx-\yy|:\xx,\yy\in\Om\}$为$\Om$的直径。记$\Phi(\xx)$为Laplace方程的基本解(见讲义上的定义3.3.1)。今固定$\xx\in\Om$, 证明:
\begin{itemize}
\item [(1)] 格林函数$G(\xx,\yy):=\Phi(\xx-\yy)-\psi^\xx(\yy)$是唯一的,且满足$\ipo \frac{\p G}{\p N}(\xx,\yy)\dS_\yy=-1$;
\item [(2)] 当 $d\geq 3$ 时, 对任意$\yy\neq \xx,~\yy\in\Om$有$0<G(\xx,\yy)<\Phi(\xx-\yy)$成立;
\item [(3)] 当 $d=2$ 时, 对任意$\yy\neq \xx,~\yy\in\Om$有$0<G(\xx,\yy)<-\frac{1}{2\pi}\ln\frac{|\xx-\yy|}{D}$成立.
\end{itemize}

提示:(1)边值为1且在有界区域$\Om$中调和的函数只能是常值函数1. (2)-(3)思考怎么用极大值原理,本题(2)是以前的考试原题。
\end{problem}

\begin{problem}[习题5.2.3,球上的Harnack不等式]
设$u$是闭球$\overline{B(\mathbf{0},R)}\subset \R^d$上的非负调和函数。
\begin{itemize}
\item [(1)] 结合讲义上的公式(5.2.9)证明
\[
R^{d-2}\frac{R-|\xx|}{(R+|\xx|)^{d-1}}u(\mathbf{0})\leq u(\xx)\leq R^{d-2}\frac{R+|\xx|}{(R-|\xx|)^{d-1}}u(\mathbf{0})
\]
\item [(2)] 用(1)证明:$\forall r\in(0,R)$, 成立不等式$\sup\limits_{B(\mathbf{0},r)}u\leq (\frac{R+r}{R-r})^d\inf\limits_{B(\mathbf{0},r)}u$. 
\end{itemize}
\end{problem}

\begin{problem}[问题4.1.1,选做]
习题4.1.4中,假设初值$u(0,x)=u_0(x)\in C^1([0,\pi])$且满足$u_0(0)=u_0(\pi)=0$,证明:存在常数$C>0$,使得对应的解满足$\sup\limits_{x\in[0,\pi]}|u(t,x)|\leq Ce^{-t}.$ 

提示:将$u_0(x)$与习题4.1.4(1)初值的常数倍相比较,这是本题假设初值$C^1$的原因所在。
\end{problem}
\end{document}
