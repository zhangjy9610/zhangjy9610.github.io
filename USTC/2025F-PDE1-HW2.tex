\documentclass[UTF8,fontset=windows,12pt]{ctexart}
%\setcounter{tocdepth}{1}
%\setcounter{secnumdepth}{4}
%\usepackage{ctex} %若要输入中文请取消注释或者把article改成ctexart
% This first part of the file is called the PREAMBLE. It includes
% customizations and command definitions. The preamble is everything
% between \documentclass and \begin{document}.
%\usepackage[OT1]{fontenc}
%\usepackage{fontspec}
%\setmainfont{Times New Roman}
%\setCJKmainfont{Microsoft YaHei}
\usepackage[left=1.91cm,right=1.91cm,top=2.54cm,bottom=2.54cm]{geometry}
\usepackage{amsfonts}
\usepackage{amsmath}
\usepackage{amssymb}
\usepackage[upint,lcgreekalpha]{stix}
\usepackage[type1]{garamondlibre}
%\usepackage{esint}
\usepackage{pgfplots}
\pgfplotsset{compat=1.18}
\usepackage{tikz}
\usetikzlibrary{arrows}
\usepackage{extarrows}
\usepackage[only,llbracket,rrbracket]{stmaryrd}
\usepackage{titlesec}
\usepackage{bm}
\usepackage{graphicx}
\usepackage{appendix}
\usepackage{tikz-cd}
\usepackage{tikz-3dplot}
\usetikzlibrary{arrows.meta, decorations.pathreplacing, calc, patterns}
\usepackage{float}
\usepackage{mathtools}
\usepackage{amsthm}
\usepackage{verbatim}              % better theorem environments
\usepackage{setspace}
\usepackage{enumitem}
\setenumerate[1]{itemsep=0pt,partopsep=0pt,parsep=\parskip,topsep=5pt}
\setitemize[1]{itemsep=0pt,partopsep=0pt,parsep=\parskip,topsep=5pt}
\setdescription{itemsep=0pt,partopsep=0pt,parsep=\parskip,topsep=5pt}
\usepackage{xcolor}
\usepackage[colorlinks,linkcolor=black,anchorcolor=black,citecolor=black]{hyperref}
%\usepackage{apacite}
\usepackage[numbers]{natbib}
%\usepackage{showkeys}
\usepackage{cases}
\usepackage{fancyhdr}
\usepackage[scaled=0.92]{helvet}	% set Helvetica as the sans-serif font
%\usepackage{upgreek}
%\renewcommand{\rmdefault}{ptm}		% set Times as the default text font
%\usepackage{newtxtext}
\bibliographystyle{plain}
% If AMS-LaTeX is used, it can be loaded before or after mtpro2		
% The following loads metro and defines some common MTPro options [2, 4]
%\usepackage[lite,subscriptcorrection,nofontinfo]{mtpro2}
%\pdfmapfile{+mtpro2.map}

%\usepackage{txfonts}

% various theorems, numbered by section

\makeatletter
\renewenvironment{proof}[1][\proofname]{\par
  \pushQED{\qed}%
  \normalfont \topsep6\p@\@plus6\p@\relax
  \trivlist
  \item[\hskip\labelsep
        \bfseries % 关键修改:斜体改为加粗
        #1\@addpunct{.}]\ignorespaces
}{%
  \popQED\endtrivlist\@endpefalse
}
\makeatother
\newtheoremstyle{kaiti-theorem}   % 样式名称
  {3pt}                           % 上方间距
  {3pt}                           % 下方间距
  {\kaishu\upshape}               % 正文字体:中文楷体 + 英文直立
  {}                              % 缩进量
  {\bfseries\upshape}      % 头部字体(定理名):中文楷体粗体 + 英文直立粗体
  {.}                             % 定理名后标点
  {0.5em}                         % 定理名后间距
  {}                              % 头部说明(留空)

% 应用自定义样式
\theoremstyle{kaiti-theorem}
\newtheorem{thm}{定理}[section]
\newtheorem{nota}[thm]{记号}
\newtheorem{question}{问题}
\newtheorem*{questionn}{思考}
\newtheorem{exe}{习题}[section]
\newtheorem{assump}[thm]{假设}
\newtheorem*{thmm}{定理}
\newtheorem{defn}{定义}[section]
\newtheorem{lem}[thm]{引理}
\newtheorem*{lemm}{引理}
\newtheorem{prop}[thm]{命题}
\newtheorem*{propp}{命题}
\newtheorem*{claim}{断言}
\newtheorem{cor}[thm]{推论}
\newtheorem{conj}[thm]{猜想}
\newtheorem{rmk}{注记}[section]
\newtheorem{ex}{例}[section]
\newtheorem{pf}{证明}
\newtheorem{problem}{作业题}
\numberwithin{equation}{section}





\DeclareMathOperator{\id}{id}
\DeclareMathOperator*{\esssup}{ess\,sup}
\renewcommand{\Re}{\textbf{Re}\,}  
\renewcommand{\Im}{\textbf{Im}\,}  
\newcommand{\R}{\mathbb{R}}  
\newcommand{\FH}{\mathbb{H}}  
\newcommand{\C}{\mathbb{C}}  
\newcommand{\Z}{\mathbb{Z}}
\newcommand{\X}{\mathbb{X}}
\newcommand{\N}{\mathbb{N}}
\newcommand{\Q}{\mathbb{Q}}
\newcommand{\T}{\mathbb{T}}
\newcommand{\TT}{\mathcal{T}}
\newcommand{\TP}{\overline{\partial}{}}
\newcommand{\TJ}{\langle\TP\rangle}
\newcommand{\TL}{\overline{\Delta}{}}
\newcommand{\curl}{\text{curl }}
\newcommand{\dive}{\text{div }}
\newcommand{\bd}[1]{\mathbf{#1}}  % for bolding symbols
\newcommand{\bds}[1]{\boldsymbol{#1}}  % for bolding symbols
\newcommand{\RR}{\mathcal{R}}      % for Real numbers
\newcommand{\sh}{\mathcal{S}}  
\newcommand{\sph}{\mathbb{S}}  
\newcommand{\Om}{\Omega}
\newcommand{\OmT}{{\Omega_T}}
\newcommand{\ut}{{U_T}}
\newcommand{\vp}{\varphi}
\newcommand{\Th}{\Theta}
\newcommand{\Lam}{\Lambda}
\newcommand{\Omb}{{\overline{\Omega}}}
\newcommand{\OmbT}{{\overline{\Omega_T}}}
\newcommand{\ub}{{\overline{U}}}
\newcommand{\ubt}{{\overline{U_T}}}
\newcommand{\q}{\quad}
\newcommand{\p}{\partial}
\newcommand{\pb}{\boldsymbol{\partial}}
\newcommand{\lap}{\Delta}
\newcommand{\dd}{\mathfrak{D}}
\newcommand{\DD}{\mathcal{D}}
\newcommand{\den}{{\bar\rho}_{0}}
\newcommand{\Dt}{{\p}_{t}+v^{k}{\p}_{k}}
\newcommand{\col}[1]{\left[\begin{matrix} #1 \end{matrix} \right]}
\newcommand{\noi}{\noindent}
\newcommand{\nab}{\nabla}
\newcommand{\cnab}{\overline{\nab}}
\newcommand{\cp}{\overline{\partial}{}}
\newcommand{\dx}{\,\mathrm{d}x}
\newcommand{\dxx}{\,\mathrm{d}\bds{x}}
\newcommand{\xxi}{\boldsymbol{\xi}}
\newcommand{\eeta}{\boldsymbol{\eta}}
\newcommand{\dxxi}{\,\mathrm{d}\boldsymbol{\xi}}
\newcommand{\deeta}{\,\mathrm{d}\boldsymbol{\eta}}
\newcommand{\dxi}{\,\mathrm{d}\xi}
\newcommand{\deta}{\,\mathrm{d}\eta}
\newcommand{\dy}{\,\mathrm{d}y}
\newcommand{\dyy}{\,\mathrm{d}\bds{y}}
\newcommand{\dz}{\,\mathrm{d}z}
\newcommand{\dzz}{\,\mathrm{d}\bds{z}}
\newcommand{\dph}{\,\mathrm{d}\varphi}
\newcommand{\dt}{\,\mathrm{d}t}
\newcommand{\dr}{\,\mathrm{d}r}
\newcommand{\dtt}{\,\mathrm{d}\tau}
\newcommand{\dS}{\,\mathrm{d}S}
\newcommand{\dss}{\,\mathrm{d}\sigma}
\newcommand{\dmu}{\,\mathrm{d}\mu}
\newcommand{\dnu}{\,\mathrm{d}\nu}
\newcommand{\dV}{\,\mathrm{d}V}
\newcommand{\ds}{\,\mathrm{d}s}
\newcommand{\drr}{\,\mathrm{d}\rho}
\newcommand{\dist}{\text{dist }}
\newcommand{\vol}{\text{vol}\,}
\newcommand{\volo}{\text{vol}(\Omega)}
\newcommand{\spt}{\text{Spt}\,}
\newcommand{\Tr}{\text{Tr}\,}
\newcommand{\lee}{\langle}
\newcommand{\ree}{\rangle}
\newcommand{\xee}{\langle\xi\rangle}
\newcommand{\xxee}{\langle\boldsymbol{\xi}\rangle}
\newcommand{\xeta}{\langle\eta\rangle}
\newcommand{\xeeta}{\langle\boldsymbol{\eta}\rangle}
\newcommand{\kk}{\kappa}
\newcommand{\lam}{\lambda}
\newcommand{\io}{\int_{\Omega}}
\newcommand{\iu}{\int_{U}}
\newcommand{\is}{\int_{\Sigma}}
\newcommand{\ipo}{\int_{\partial\Omega}}
\newcommand{\ipu}{\int_{\partial U}}
\newcommand{\ig}{\int_{\Gamma}}
\newcommand{\ir}{\int_{\R}}
\newcommand{\ird}{\int_{\R^d}}
\newcommand{\irn}{\int_{\R^n}}
\newcommand{\fpi}{\frac{1}{(2\pi)^{\frac{d}{2}}}}
\newcommand{\ddt}{\frac{\mathrm{d}}{\mathrm{d}t}}
\newcommand{\ddx}{\frac{\mathrm{d}}{\mathrm{d}x}}
\newcommand{\eps}{\varepsilon}
\providecommand{\jump}[1]{\left\llbracket #1 \right\rrbracket }
\providecommand{\len}[1]{\lee #1 \ree }
\providecommand{\ino}[1]{\left\| #1 \right\| }
\providecommand{\bno}[1]{\left| #1 \right| }
\newcommand{\red}{\textcolor{red}}
\newcommand{\green}{\textcolor{green}}
\newcommand{\blue}{\textcolor{blue}}
\newcommand{\nnr}{{[n+1]}}
\newcommand{\nnn}{{[n]}}
\newcommand{\nnl}{{[n-1]}}
\newcommand{\nnll}{{[n-2]}}	
\newcommand{\mmr}{{[m+1]}}
\newcommand{\mmm}{{[m]}}
\newcommand{\mml}{{[m-1]}}
\newcommand{\mmll}{{[m-2]}}

%---------------Special variables--------------
\newcommand{\CC}{\mathfrak{C}}  
\newcommand{\NN}{\mathbf{N}}
\newcommand{\LH}{\mathcal{L}}
\newcommand{\HH}{\mathcal{H}}  
\newcommand{\EE}{\mathcal{E}}  
\newcommand{\FT}{\mathcal{F}}  
\newcommand{\fT}{\mathbf{T}}  
\newcommand{\FA}{\mathbf{A}}  
\newcommand{\MH}{\mathcal{M}}
\newcommand{\NH}{\mathcal{N}}
\newcommand{\PH}{\mathcal{P}}
\newcommand{\VH}{\mathcal{V}}
\newcommand{\XX}{\mathfrak{X}}
\newcommand{\xx}{{\bds{x}}}
\newcommand{\bx}{\mathbf{x}}
\newcommand{\by}{\mathbf{y}}
\newcommand{\bp}{\mathbf{p}}
\newcommand{\bq}{\mathbf{q}}
\newcommand{\pp}{{\bds{p}}}
\newcommand{\qq}{{\bds{q}}}
\newcommand{\uu}{\mathbf{u}}
\newcommand{\ww}{\mathbf{w}}
\newcommand{\vv}{\mathbf{v}}
\newcommand{\yy}{{\bds{y}}}
\newcommand{\zz}{{\bds{z}}}
\newcommand{\zo}{\mathbf{0}}
\newcommand{\ee}{\mathbf{e}}
\newcommand{\fa}{\mathbf{a}}
\newcommand{\fb}{\mathbf{b}}
\newcommand{\fc}{\mathbf{c}}
\newcommand{\ff}{\bds{f}}
\newcommand{\fr}{\mathbf{r}}
\newcommand{\fh}{\mathbf{h}}
\newcommand{\fm}{\mathbf{m}}
\newcommand{\fg}{\mathbf{g}}
\newcommand{\FF}{\mathbf{F}}
\newcommand{\GG}{\mathbf{G}}
\newcommand{\BB}{\mathbf{B}}



\begin{document}
\title{\LARGE{\bf  2025年秋季学期偏微分方程作业二}}
\date{截止时间:2025年12月8日下课前}

\maketitle

\textbf{作业题1--6是必做题,作业题7--8选做。所有作业中的选做题、以及讲义上没布置的“问题”在本课程中都不作要求。}

\begin{problem}[习题2.4.2]
本题考虑半平面内的调和函数求解。
\begin{itemize}
\item [(1)] 给定$y>0$, 计算$F_y(\xi)=e^{-|\xi|y}$关于频率变量$\xi\in\R$的傅立叶逆变换。
\item [(2)] 计算如下边值问题\underline{有界解}的显式表达式
\[
\begin{cases}
u_{xx}+u_{yy}=0&~~~x\in\R,~y>0,\\
u(x,0)=\varphi(x)&~~~x\in\R,
\end{cases}
\]其中$\varphi\in\mathcal{S}(\R)$是给定的。
\item [(3)] 证明:对任意$y>0$有$\int_\R|u(x,y)|\dx\leqslant \int_\R|\varphi(x)|\dx$.
\end{itemize}
\end{problem}

\begin{problem}[习题2.4.3, Cole-Hopf变换]
设$a>0,b\in\R$是给定的常数,求解如下拟线性热方程。
\[
\begin{cases}
\p_t u-a^2\lap u+b|\nab u|^2=0&~~t>0,~\xx\in\R^d\\
u(0,\xx)=\varphi(\xx)&~~t=0,~\xx\in\R^d.
\end{cases}
\] 

提示:考虑$v=e^{-\frac{bu}{a^2}}$满足的方程。
\end{problem}



\begin{problem}[习题2.4.4]
求解如下粘性Burgers方程
\[
\begin{cases}
u_t-a^2u_{xx}+uu_x=0&~~t>0,~x\in\R\\
u(0,x)=\varphi(x)&~~t=0,~x\in\R.
\end{cases}
\]这里$a\in\R$是给定的常数。(提示:考虑$v(t,x)=\int_{-\infty}^x u(t,y)\dy$满足的方程。)
\end{problem}

\begin{problem}[习题2.4.8]
考虑齐次热传导方程的初值问题
\[
\p_t u-k\lap u=0\q\text{ in }\R_+\times\R^d,\q\q u(0,\xx)=\vp(\xx)\q(\xx\in\R^d).
\]设初值$\vp\in\sh(\R^d)$, 由泊松公式给出的解为$u(t,\xx)$. 
\begin{itemize}
\item [(1)] 证明:存在常数$C>0$, 使得$|u(t,\xx)-\vp(\xx)|\leqslant C\sqrt{kt}$对任意$t>0,\xx\in\R^d$成立。
\item [(2)] 证明:存在常数$C>0$, 使得$\sup\limits_{\xx\in\R^d}|u(t,\xx)|\leqslant Ct^{-\frac{d}{4}}\|\vp\|_{L^2(\R^d)}$对任意$t>0$成立。
\end{itemize}

提示:(1) 注意$\ird K(t,\yy)\dyy=1$, 将$\vp(\xx)$写作$\ird K(t,\yy)\vp(\xx)\dyy$. (2) 将$u$写成$(\hat{u})^\vee$然后用 傅立叶逆变换的定义得到$|u(t,\xx)|\leqslant \ird |\hat{u}(t,\xxi)|\dxxi$, 代入$\hat{u}(t,\xxi)$的表达式后用Cauchy-Schwarz不等式。
\end{problem}


\begin{problem}[问题C.1.2]
给定点$\xx_0,\xxi_0\in \R^d$ 以及函数 $f\in\mathcal{S}(\R^d)$, 证明如下海森堡不确定性原理:
\begin{equation}
\left(\ird |(\xx-\xx_0)f(\xx)|^2\dxx\right)\left(\ird |(\xxi-\xxi_0)\hat{f}(\xxi)|^2\dxxi\right)\geqslant \frac{d^2}{4} \left(\ird |f(\xx)|^2\dxx\right)^2.
\end{equation} 这个不等式表明动量和位置不可能在给定的动量$\xxi_0$和给定的位置$\xi_0$附近被同时确定。

提示:只需证明$\xxi_0=\xx_0=\bd{0}$的情况即可,否则考虑$g(\xx)=f(\xx+\xx_0)e^{-i\xx\cdot\xxi_0}$并利用命题C.1.3(2)约化到这一特殊情况。利用Plancherel恒等式可得 $|\xxi\hat{f}(\xxi)|^2=|\widehat{\nab f}(\xxi)|^2$,之后再用Plancherel恒等式和 Cauchy-Schwarz不等式证明左边$\geqslant (\ird |(\xx\cdot\nab f)f|\dxx)^2$, 最后用$(\nab f)f=\frac12 \nab(f^2)$, 然后分部积分一次。
\end{problem}

\begin{problem}[问题2.4.2]
设$\Om\subset\R^d$是有界区域且边界光滑,$u(t,\xx)\in C_1^2((0,\infty)\times \Om)\cap C([0,\infty)\times\overline{\Om})$ 是如下方程的解
\begin{equation*}
\begin{cases}
\p_t u-\lap u=0&~~t>0,~\xx\in \Om\\
u(0,\xx)=u_0(\xx)&~~\xx\in\overline{\Om}\\
u=0&~~t\geqslant 0,~\xx\in\p \Om,
\end{cases}
\end{equation*} 其中 $u_0\in C^2(\overline{\Om})$给定。 证明:存在常数$a>0$(仅和区域本身有关)使得
$$\io (u(t,\xx))^2\dxx\leqslant e^{-at}\io (u_0(\xx))^2\dxx.$$

提示:对有界区域$\Om$, 若$f|_{\p\Om}=0$, 则必存在常数$C>0$使得$\io f^2\dxx\leqslant C\io |\nab f|^2\dxx$. 思考如何利用类似于“微积分基本定理”的想法去证明这个事实。
\end{problem}



\begin{problem}[习题2.5.1,选做]
考虑$\R^d$中的Klein-Gordon方程的初值问题
\begin{equation*}
\begin{cases}
\p_t^2u-\lap u+m^2 u=0~~~&t>0,~\xx\in\R^d\\
u(0,\xx)=\varphi(\xx),~\p_t u(0,\xx)=\psi(\xx)~~~&\xx\in\R^d,
\end{cases}
\end{equation*}其中 $\varphi,\psi\in\mathcal{S}(\R^d)$.
\begin{itemize}
\item [(1)] 定义能量 $E(t)=\frac12\int_{\R^d}(\p_t u)^2+|\nab u|^2+m^2 u^2\dxx.$ 证明:$E(t)$是守恒量。
\item [(2)] 用$\hat{\varphi}$和$\hat{\psi}$写出解的傅立叶变换$\hat{u}(\xxi)$的表达式。
\item [(3)] 证明:$\lim\limits_{t\to+\infty}\int_{\R^d}|\nab u|^2+m^2u^2\dxx=E(0)$,进而该方程也有能量渐近均分原理$$\lim\limits_{t\to+\infty}\frac12\int_{\R^d}|\nab u|^2+m^2u^2\dxx=\lim\limits_{t\to+\infty}\frac12\int_{\R^d}(\p_t u)^2\dxx.$$
\end{itemize}

提示:本题基本可如法炮制波动方程的证明,但请思考如何正确使用Riemann-Lebesgue引理。
\end{problem}


\begin{problem}[问题C.1.7,选做,{$L^\infty$}空间的Sobolev嵌入定理]
对$s\in\R,~f\in \sh(\R^d)$, 我们定义范数$\|f\|_{s}:=\|\xxee^s\hat{f}(\xxi)\|_{L^2(\R^d)}$, 其中$\xxee:=\sqrt{1+|\xxi|^2}$. 今假设$s>\frac{d}{2}$, 证明:
\begin{itemize}
\item [(1)] 存在常数$C>0$使得对任意$f\in \sh(\R^d)$成立$\max\limits_{\R^d}|f|\leqslant C\|f\|_{s}$.
\item [(2)] 存在常数$C>0$使得对任意$f,g\in \sh(\R^d)$成立$\|fg\|_{s}\leqslant C\|f\|_{s}\|g\|_{s}$.
\end{itemize}

提示:(1)将$f$写成$f=(\hat{f})^\vee$并用傅立叶逆变换的定义写成积分,然后乘以$\xxee^s$和$\xxee^{-s}$, 然后利用$s>\frac{d}{2}$得出$\xxee^{-s}\in L^2(\R^d)$, 用Cauchy-Schwarz不等式即得。(2) 注意到$\widehat{fg}=(2\pi)^{-\frac{d}{2}}(\hat{f}*\hat{g})$, 把卷积写成积分式并利用$\xxee^s\leqslant C(\langle\xxi-\eeta\rangle^s+\xeeta^s)$即可。
\end{problem}


\end{document}
