\documentclass[UTF8,fontset=windows,12pt]{ctexart}
%\setcounter{tocdepth}{1}
%\setcounter{secnumdepth}{4}
%\usepackage{ctex} %若要输入中文请取消注释或者把article改成ctexart
% This first part of the file is called the PREAMBLE. It includes
% customizations and command definitions. The preamble is everything
% between \documentclass and \begin{document}.
%\usepackage[OT1]{fontenc}
%\usepackage{fontspec}
%\setmainfont{Times New Roman}
%\setCJKmainfont{Microsoft YaHei}
\usepackage[left=1.91cm,right=1.91cm,top=2.54cm,bottom=2.54cm]{geometry}
\usepackage{amsfonts}
\usepackage{amsmath}
\usepackage{amssymb}
\usepackage[upint,lcgreekalpha]{stix}
\usepackage[type1]{garamondlibre}
%\usepackage{esint}
\usepackage{pgfplots}
\pgfplotsset{compat=1.18}
\usepackage{tikz}
\usetikzlibrary{arrows}
\usepackage{extarrows}
\usepackage[only,llbracket,rrbracket]{stmaryrd}
\usepackage{titlesec}
\usepackage{bm}
\usepackage{graphicx}
\usepackage{appendix}
\usepackage{tikz-cd}
\usepackage{tikz-3dplot}
\usetikzlibrary{arrows.meta, decorations.pathreplacing, calc, patterns}
\usepackage{float}
\usepackage{mathtools}
\usepackage{amsthm}
\usepackage{verbatim}              % better theorem environments
\usepackage{setspace}
\usepackage{enumitem}
\setenumerate[1]{itemsep=0pt,partopsep=0pt,parsep=\parskip,topsep=5pt}
\setitemize[1]{itemsep=0pt,partopsep=0pt,parsep=\parskip,topsep=5pt}
\setdescription{itemsep=0pt,partopsep=0pt,parsep=\parskip,topsep=5pt}
\usepackage{xcolor}
\usepackage[colorlinks,linkcolor=black,anchorcolor=black,citecolor=black]{hyperref}
%\usepackage{apacite}
\usepackage[numbers]{natbib}
%\usepackage{showkeys}
\usepackage{cases}
\usepackage{fancyhdr}
\usepackage[scaled=0.92]{helvet}	% set Helvetica as the sans-serif font
%\usepackage{upgreek}
%\renewcommand{\rmdefault}{ptm}		% set Times as the default text font
%\usepackage{newtxtext}
\bibliographystyle{plain}
% If AMS-LaTeX is used, it can be loaded before or after mtpro2		
% The following loads metro and defines some common MTPro options [2, 4]
%\usepackage[lite,subscriptcorrection,nofontinfo]{mtpro2}
%\pdfmapfile{+mtpro2.map}

%\usepackage{txfonts}

% various theorems, numbered by section

\makeatletter
\renewenvironment{proof}[1][\proofname]{\par
  \pushQED{\qed}%
  \normalfont \topsep6\p@\@plus6\p@\relax
  \trivlist
  \item[\hskip\labelsep
        \bfseries % 关键修改:斜体改为加粗
        #1\@addpunct{.}]\ignorespaces
}{%
  \popQED\endtrivlist\@endpefalse
}
\makeatother
\newtheoremstyle{kaiti-theorem}   % 样式名称
  {3pt}                           % 上方间距
  {3pt}                           % 下方间距
  {\kaishu\upshape}               % 正文字体:中文楷体 + 英文直立
  {}                              % 缩进量
  {\bfseries\upshape}      % 头部字体(定理名):中文楷体粗体 + 英文直立粗体
  {.}                             % 定理名后标点
  {0.5em}                         % 定理名后间距
  {}                              % 头部说明(留空)

% 应用自定义样式
\theoremstyle{kaiti-theorem}
\newtheorem{thm}{定理}[section]
\newtheorem{nota}[thm]{记号}
\newtheorem{question}{问题}
\newtheorem*{questionn}{思考}
\newtheorem{exe}{习题}[section]
\newtheorem{assump}[thm]{假设}
\newtheorem*{thmm}{定理}
\newtheorem{defn}{定义}[section]
\newtheorem{lem}[thm]{引理}
\newtheorem*{lemm}{引理}
\newtheorem{prop}[thm]{命题}
\newtheorem*{propp}{命题}
\newtheorem*{claim}{断言}
\newtheorem{cor}[thm]{推论}
\newtheorem{conj}[thm]{猜想}
\newtheorem{rmk}{注记}[section]
\newtheorem{ex}{例}[section]
\newtheorem{pf}{证明}
\newtheorem{problem}{作业题}
\numberwithin{equation}{section}





\DeclareMathOperator{\id}{id}
\DeclareMathOperator*{\esssup}{ess\,sup}
\renewcommand{\Re}{\textbf{Re}\,}  
\renewcommand{\Im}{\textbf{Im}\,}  
\newcommand{\R}{\mathbb{R}}  
\newcommand{\FH}{\mathbb{H}}  
\newcommand{\C}{\mathbb{C}}  
\newcommand{\Z}{\mathbb{Z}}
\newcommand{\X}{\mathbb{X}}
\newcommand{\N}{\mathbb{N}}
\newcommand{\Q}{\mathbb{Q}}
\newcommand{\T}{\mathbb{T}}
\newcommand{\TT}{\mathcal{T}}
\newcommand{\TP}{\overline{\partial}{}}
\newcommand{\TJ}{\langle\TP\rangle}
\newcommand{\TL}{\overline{\Delta}{}}
\newcommand{\curl}{\text{curl }}
\newcommand{\dive}{\text{div }}
\newcommand{\bd}[1]{\mathbf{#1}}  % for bolding symbols
\newcommand{\bds}[1]{\boldsymbol{#1}}  % for bolding symbols
\newcommand{\RR}{\mathcal{R}}      % for Real numbers
\newcommand{\sh}{\mathcal{S}}  
\newcommand{\sph}{\mathbb{S}}  
\newcommand{\Om}{\Omega}
\newcommand{\OmT}{{\Omega_T}}
\newcommand{\ut}{{U_T}}
\newcommand{\vp}{\varphi}
\newcommand{\Th}{\Theta}
\newcommand{\Lam}{\Lambda}
\newcommand{\Omb}{{\overline{\Omega}}}
\newcommand{\OmbT}{{\overline{\Omega_T}}}
\newcommand{\ub}{{\overline{U}}}
\newcommand{\ubt}{{\overline{U_T}}}
\newcommand{\q}{\quad}
\newcommand{\p}{\partial}
\newcommand{\pb}{\boldsymbol{\partial}}
\newcommand{\lap}{\Delta}
\newcommand{\dd}{\mathfrak{D}}
\newcommand{\DD}{\mathcal{D}}
\newcommand{\den}{{\bar\rho}_{0}}
\newcommand{\Dt}{{\p}_{t}+v^{k}{\p}_{k}}
\newcommand{\col}[1]{\left[\begin{matrix} #1 \end{matrix} \right]}
\newcommand{\noi}{\noindent}
\newcommand{\nab}{\nabla}
\newcommand{\cnab}{\overline{\nab}}
\newcommand{\cp}{\overline{\partial}{}}
\newcommand{\dx}{\,\mathrm{d}x}
\newcommand{\dxx}{\,\mathrm{d}\bds{x}}
\newcommand{\xxi}{\boldsymbol{\xi}}
\newcommand{\eeta}{\boldsymbol{\eta}}
\newcommand{\dxxi}{\,\mathrm{d}\boldsymbol{\xi}}
\newcommand{\deeta}{\,\mathrm{d}\boldsymbol{\eta}}
\newcommand{\dxi}{\,\mathrm{d}\xi}
\newcommand{\deta}{\,\mathrm{d}\eta}
\newcommand{\dy}{\,\mathrm{d}y}
\newcommand{\dyy}{\,\mathrm{d}\bds{y}}
\newcommand{\dz}{\,\mathrm{d}z}
\newcommand{\dzz}{\,\mathrm{d}\bds{z}}
\newcommand{\dph}{\,\mathrm{d}\varphi}
\newcommand{\dt}{\,\mathrm{d}t}
\newcommand{\dr}{\,\mathrm{d}r}
\newcommand{\dtt}{\,\mathrm{d}\tau}
\newcommand{\dS}{\,\mathrm{d}S}
\newcommand{\dss}{\,\mathrm{d}\sigma}
\newcommand{\dmu}{\,\mathrm{d}\mu}
\newcommand{\dnu}{\,\mathrm{d}\nu}
\newcommand{\dV}{\,\mathrm{d}V}
\newcommand{\ds}{\,\mathrm{d}s}
\newcommand{\drr}{\,\mathrm{d}\rho}
\newcommand{\dist}{\text{dist }}
\newcommand{\vol}{\text{vol}\,}
\newcommand{\volo}{\text{vol}(\Omega)}
\newcommand{\spt}{\text{Spt}\,}
\newcommand{\Tr}{\text{Tr}\,}
\newcommand{\lee}{\langle}
\newcommand{\ree}{\rangle}
\newcommand{\xee}{\langle\xi\rangle}
\newcommand{\xxee}{\langle\boldsymbol{\xi}\rangle}
\newcommand{\xeta}{\langle\eta\rangle}
\newcommand{\xeeta}{\langle\boldsymbol{\eta}\rangle}
\newcommand{\kk}{\kappa}
\newcommand{\lam}{\lambda}
\newcommand{\io}{\int_{\Omega}}
\newcommand{\iu}{\int_{U}}
\newcommand{\is}{\int_{\Sigma}}
\newcommand{\ipo}{\int_{\partial\Omega}}
\newcommand{\ipu}{\int_{\partial U}}
\newcommand{\ig}{\int_{\Gamma}}
\newcommand{\ir}{\int_{\R}}
\newcommand{\ird}{\int_{\R^d}}
\newcommand{\irn}{\int_{\R^n}}
\newcommand{\fpi}{\frac{1}{(2\pi)^{\frac{d}{2}}}}
\newcommand{\ddt}{\frac{\mathrm{d}}{\mathrm{d}t}}
\newcommand{\ddx}{\frac{\mathrm{d}}{\mathrm{d}x}}
\newcommand{\eps}{\varepsilon}
\providecommand{\jump}[1]{\left\llbracket #1 \right\rrbracket }
\providecommand{\len}[1]{\lee #1 \ree }
\providecommand{\ino}[1]{\left\| #1 \right\| }
\providecommand{\bno}[1]{\left| #1 \right| }
\newcommand{\red}{\textcolor{red}}
\newcommand{\green}{\textcolor{green}}
\newcommand{\blue}{\textcolor{blue}}
\newcommand{\nnr}{{[n+1]}}
\newcommand{\nnn}{{[n]}}
\newcommand{\nnl}{{[n-1]}}
\newcommand{\nnll}{{[n-2]}}	
\newcommand{\mmr}{{[m+1]}}
\newcommand{\mmm}{{[m]}}
\newcommand{\mml}{{[m-1]}}
\newcommand{\mmll}{{[m-2]}}

%---------------Special variables--------------
\newcommand{\CC}{\mathfrak{C}}  
\newcommand{\NN}{\mathbf{N}}
\newcommand{\LH}{\mathcal{L}}
\newcommand{\HH}{\mathcal{H}}  
\newcommand{\EE}{\mathcal{E}}  
\newcommand{\FT}{\mathcal{F}}  
\newcommand{\fT}{\mathbf{T}}  
\newcommand{\FA}{\mathbf{A}}  
\newcommand{\MH}{\mathcal{M}}
\newcommand{\NH}{\mathcal{N}}
\newcommand{\PH}{\mathcal{P}}
\newcommand{\VH}{\mathcal{V}}
\newcommand{\XX}{\mathfrak{X}}
\newcommand{\xx}{{\bds{x}}}
\newcommand{\bx}{\mathbf{x}}
\newcommand{\by}{\mathbf{y}}
\newcommand{\bp}{\mathbf{p}}
\newcommand{\bq}{\mathbf{q}}
\newcommand{\pp}{{\bds{p}}}
\newcommand{\qq}{{\bds{q}}}
\newcommand{\uu}{\mathbf{u}}
\newcommand{\ww}{\mathbf{w}}
\newcommand{\vv}{\mathbf{v}}
\newcommand{\yy}{{\bds{y}}}
\newcommand{\zz}{{\bds{z}}}
\newcommand{\zo}{\mathbf{0}}
\newcommand{\ee}{\mathbf{e}}
\newcommand{\fa}{\mathbf{a}}
\newcommand{\fb}{\mathbf{b}}
\newcommand{\fc}{\mathbf{c}}
\newcommand{\ff}{\bds{f}}
\newcommand{\fr}{\mathbf{r}}
\newcommand{\fh}{\mathbf{h}}
\newcommand{\fm}{\mathbf{m}}
\newcommand{\fg}{\mathbf{g}}
\newcommand{\FF}{\mathbf{F}}
\newcommand{\GG}{\mathbf{G}}
\newcommand{\BB}{\mathbf{B}}



\begin{document}
\title{\LARGE{\bf  2025年秋季学期偏微分方程作业三}}
\author{分离变量法}
\date{截止时间:2025年12月22日下课前}

\maketitle

\textbf{作业题1--8是必做题,作业题9选做。}

\begin{problem}[习题3.1.2]
考虑一维带阻尼的波动方程,其中常数$d\in(0,2)$
\[
\begin{cases}
u_{tt}-u_{xx}+du_t=0&~~~t>0,~0<x<\pi;\\
u(t,0)=u(t,\pi)=0&~~~t\geq 0.
\end{cases}
\]
求该方程具有分离形式的解$u(t,x)=T(t)X(x)$, 并说明$t\to+\infty$时这些解的行为。
\end{problem}

\begin{problem}[习题3.1.5]
设$A,B$是常数,求解如下方程
\[
\begin{cases}
u_{tt}-u_{xx}=0&~~~t>0,~0<x<\pi\\
u(0,x)=0,~u_t(0,x)=0&~~~0\leq x\leq \pi\\
u_x(t,0)=At,~~u_x(t,\pi)=Bt&~~~t\geq 0.
\end{cases}
\]
\end{problem}



\begin{problem}[习题3.1.6]
考虑具有固定端点且长度有限的弦发生的受迫振动方程
\[
\begin{cases}
u_{tt}-u_{xx}=\cos x\cos 5x \sin(\omega t)&~~t>0,~x\in(0,\pi)\\
u(0,x)=0,~u_t(0,x)=0&~~x\in [0,\pi]\\
u_x(t,0)=u_x(t,\pi)=0&~~t\geq 0,
\end{cases}
\]其中 $\omega>0$ 是常数。求解方程并讨论 $\omega$ 为何值时方程的解一致有界?即$\sup\limits_{t>0,x\in(0,\pi)}|u(t,x)|< \infty$.
\end{problem}

\begin{problem}[习题3.2.3]
考虑具有Neumann边界条件的热传导方程,其中$k\geq0$是常数
\[
\begin{cases}
u_{t}-u_{xx}=k(1-u)&~~~t>0,0<x<\pi\\
u(0,x)=\varphi(x)&~~~0\leq x\leq \pi\\
u_x(t,0)=u_x(t,\pi)=0&~~~t\geq 0.
\end{cases}
\]求解这个方程,并计算$\lim\limits_{t\to+\infty} u(t,x)$. (提示:$k=0$和$k>0$答案不同。)
\end{problem}


\begin{problem}[习题3.2.5]
考虑如下热传导方程的初边值问题
\[
u_t-u_{xx}=0~t>0,x\in(0,\pi);\q u(0,x)=\vp(x)\in C^2([0,\pi])~0\leq x\leq \pi;\q u(t,0)=u(t,\pi)=0,~t\geq 0.
\]
\begin{itemize}
\item [(1)] 证明:方程的解满足估计$\int_0^\pi u(t,x)^2\dx\leq e^{-2t}\int_0^\pi \vp(x)^2\dx.$
\item [(2)] 证明:存在常数$C>0$使得$|u(t,x)|\leq Ce^{-t}$对任意$t>0,x\in[0,\pi]$成立。
\end{itemize}
提示:无论使用能量法还是分离变量法,可能都需要使用Parseval恒等式;对(2),思考为什么这里假设了$\vp\in C^2([0,\pi])$,具有该光滑性的函数的傅立叶系数具有怎样的阶?
\end{problem}

\begin{problem}[习题3.3.1]
设$\Om\subset\R^2$是单位圆盘。求解方程$\lap u=2\text{ in }\Om,~u|_{\p\Om}=2x_1x_2.$
\end{problem}

\begin{problem}[习题3.3.4]
证明方程$\lap u=0,~~\xx\in\R^d$的解是旋转不变的,即对$d\times d$正交方阵$\mathbf{O}$, 令$v(\xx):=u(\mathbf{O}\xx)$, 则必有$\lap v=0$.
\end{problem}

\begin{problem}[习题3.4.6,Dirichlet原理]
设$\Om\subset\R^d$是边界$C^1$的有界开集,给定函数$f\in C(\Omb), g\in C(\p\Om)$, 定义能量泛函$I[w]:=\io \frac12|\nab w|^2 -wf\dxx$, 其中$w\in\mathcal{A}:=\{w\in C^2(\Omb): w=g\text{ on }\p \Om\}.$ 
证明:$u$是$I[\cdot]$在$\mathcal{A}$上的极小化子(即$I[u]=\inf\limits_{w\in\mathcal{A}}I[w],~u\in\mathcal{A}$)当且仅当$u$是位势方程的解\[
-\lap u=f\text{ in }\Om\q\q u=g\text{ on }\p\Om.
\]

提示:本题跟特征值理论没什么关系;令$j(\eps)=I[u+\eps v]$, 其中$v\in C_c^\infty(\Om)$, 然后计算$j'(0)=0$.
\end{problem}


\begin{problem}[选做]
考虑多孔介质流方程$\p_t u-\lap(u^\gamma)=0,~(t>0,\xx\in\R^d)$. 其中假设$\gamma>1$是常数,方程的解$u\geq 0$. 今假设方程的解具有变量分离形式$u(t,\xx)=v(t)w(\xx)~(t\geq 0,\xx\in\R^d)$.
\begin{itemize}
\item [(1)] 证明:存在常数$\mu,\lam\in\R$, 使得$v(t)=((1-\gamma)\mu t+\lambda)^{\frac{1}{1-\gamma}}$.
\item [(2)] 若$w(\xx)=|\xx|^a$, 请计算$a$与$\gamma$之间应该满足怎样的关系式。
\item [(3)] 若将(1)中的$\lambda$取为正数,证明:对由(1), (2)求得的解$u(t,\xx)$, 存在时间$T_*<\infty$, 使得$u(t,\xx)$在$\xx\neq \bd{0}$, $t\to T_*$时发生爆破。
\end{itemize}
\end{problem}


\end{document}
